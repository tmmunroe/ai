\documentclass{article}
\usepackage[utf8]{inputenc}
\usepackage[margin=0.5in]{geometry}
\usepackage{graphicx}

\title{COMS 4701 - Homework 5 - Conceptual}
\author
{
Turner Mandeville
\and tmm2169
}
\date{}

\begin{document}
    \maketitle
    \section*{Question 1 - Perceptron}
    \begin{enumerate}
        \item Implement majority voting for four voters - E1, E2, E3, and E4. Draw the network and the weights. 
        \newline
        Weights should be [1,1,1,1] and the bias would be -2
        \newline
        The inputs will be [E1, E2, E3, E4], each of which is from the set $\{0, 1\}$. The outputs will also be from the set $\{0, 1\}$.
        \newline
        The weights will need to give a positive number if a majority of the voters give a 1 (Email), and a negative number or 0 if the majority of voters give a 0 (Spam). In the case of a tie, the function should return a 0, indicating that the email is Spam.
        \newline
        The weights [1,1,1,1] and bias or -2 will accomplish this:
        \newline
        \newline
        The dot product of the voter inputs and the weights will give:
        $-2 + \sum w_i * E_i$
        \newline
        \newline
        In the case of a majority voting for Email, we would have 3 or more voters saying Email (1):
        \newline
        $f(E) >= -2 + ( 1*1 + 1*1 + 1*1 + 0*1 ) = 1 > 0$
        \newline
        \newline
        In the case of a majority voting for Spam or a tie, we would have 2 or more voters saying Spam (0):
        \newline
        $f(E) <= -2 + ( 1*1 + 1*1 + 0*1 + 0*1 ) = 0 <= 0$
        \newline
        \newline
        The tree would be roughly:
        \newline
        1 --- $w_0 = -2$ --
        \newline
        E1 --- $w_1 = 1$ ----
        \newline
        E2 --- $w_2 = 1$ ------- $f(E) = -2 + \sum w_i * E_i$ --- $Step(f(E)) = 0$ if $f(E) <= 0$, otherwise 1
        \newline
        E3 --- $w_3 = 1$ ----
        \newline
        E4 --- $w_4 = 1$ --
        \newline
        \item How can you adapt the perceptron structure and weights to n experts?
        \newline
        The perceptron network for majority voting could be adapted to n experts by making the bias a function of n and leaving the rest of the weights equal to 1:
        \newline
        $bias = floor(\frac{-n}{2})$
        \newline
        This works because the output of the perceptron depends on the relationship: $bias < \sum w_i * E_i$. If the bias in this formula is set to be the number of votes that would give a tie, then in order for the inequality to hold, we would need a simple majority. And in that case, $f(E) = -bias + \sum w_i * E_i $ would be positive. If we had a minority, then the following would hold $bias >= \sum w_i * E_i$ and $f(E) = -bias + \sum w_i * E_i $ would be 0 or negative.
        \item If all voters use different information and vote with an accuracy level of 70\%. Can majority voting improve accuracy in this case on the same set of examples?
        \newline
        Yes. Since the voters use different information to make their decisions, we can conclude that the probabilities of each one being correct are effectively independent. 
        \newline
        Condorcet's Jury theorem addressed this exact type of scenario. It shows that if each voter's accuracy is $>0.5$, and voters make independent decisions, then a majority vote will give a better probability of being correct than one individual's vote alone.
    \end{enumerate}
    
    \newpage
    \section*{Question 2 - Logical Agents}
    \begin{enumerate}
        \item Use inference to prove the following rule:
        \newline
        \newline
        Part 1
        \newline
        \newline
        $\frac{\alpha \land \beta => \gamma \hspace{0.5cm}\lnot(\lnot \beta \lor \gamma}{\lnot \alpha}$
        \newline
        \newline
        $\frac{\lnot(\lnot \beta \lor \gamma)}{\lnot \lnot \beta \land \lnot \gamma}$ De-Morgan's Law
        \newline
        \newline
        $\frac{\lnot \lnot \beta \land \lnot \gamma}{\beta \land \lnot \gamma}$ Simplification
        \newline
        \newline
        Part 2
        \newline
        \newline
        $\frac{\alpha \land \beta => \gamma}{\lnot \gamma => \lnot (\alpha \land \beta)}$ Contrapositive
        \newline
        \newline
        $\frac{\lnot \gamma => \lnot (\alpha \land \beta)}{\lnot \gamma => \lnot \alpha \lor \lnot \beta}$ De-Morgan's Law
        \newline
        \newline
        From Part 1 and Part 2
        \newline
        \newline
        $\frac{\beta \land \lnot \gamma \hspace{0.5cm} \lnot \gamma => \lnot \alpha \lor \lnot \beta} {\lnot \alpha \lor \lnot \beta}$ Implication
        \newline
        \newline
        $\frac{\beta \land \lnot \gamma} {\beta}$ Simplification
        \newline
        \newline
        $\frac{\beta \hspace{0.5cm} \lnot \alpha \lor \lnot \beta}{\lnot \alpha}$
        \newline
        \item Given $KB = \{p\lor \lnot q, p \lor \lnot r, q \lor \lnot p, r \}$
        \begin{enumerate}
            \item Using resolution for propositional logic, does KB imply $\lnot p$?
            \newline
            No:
            \newline
            \newline
            $\frac{r \hspace{0.5cm} p \lor \lnot r} { p }$
            \item Use model checking to confirm whether KB implies $\lnot p$ or not?
            \newline
            \newline
            I'll just show the cases for $\lnot p$ since showing that all of these do not hold will show that the implication is false
            \newline
            \newline
            \begin{center}
             \begin{tabular}{||c | c | c || c | c | c | c || c || } 
             \hline
             p & q & r & $p\lor \lnot q$ & $p\lor \lnot r$ & $q\lor \lnot p$ & r & KB \\ [0.5ex] 
             \hline\hline
             F & F & F & T & T & T & F & F \\ 
             \hline
             F & F & T & T & F & T & T & F \\ 
             \hline
             F & T & T & F & F & T & T & F \\ 
             \hline
             F & T & F & F & T & T & F & F \\ 
             \hline
             \hline
            \end{tabular}
            \end{center}
            \newline
            \newline
            This shows that for all cases of $\lnot p$, KB is false. So the implication does not hold.
        \end{enumerate}
        
        \item The CNF of $(p \land \lnot q) \lor (p \land r)$ is 
        \newline
        $(p) \land (p \lor r) \land (\lnot q \lor p) \land (\lnot q \lor r)$
        \newline
        \newline
        Work:
        \newline
        $\frac{(p \land \lnot q) \lor (p \land r)} {(p \lor (p \land r)) \land (\lnot q \lor (p \land r))}$ Distribution of or over and
        \newline
        \newline
        $\frac{(p \lor (p \land r)) \land (\lnot q \lor (p \land r))}{((p \lor p) \land (p \lor r)) \land ((\lnot q \lor p) \land (\lnot q \lor r))} $ Distribution of and over or
        \newline
        \newline
        $\frac{((p \lor p) \land (p \lor r)) \land ((\lnot q \lor p) \land (\lnot q \lor r))}{(p) \land (p \lor r) \land (\lnot q \lor p) \land (\lnot q \lor r)} $ Simplification
        \newline
    \end{enumerate}
    \newpage
    \section*{Question 3 - Bayes Nets}
    \begin{enumerate}
        \item Write the full join distribution over the variables S, C, B.
        \newline
        \newline
        \begin{center}
         \begin{tabular}{||c | c | c || c || c ||} 
         \hline
         Smoking & Cancer & Bronchitis & Formula & P \\ [0.5ex] 
         \hline\hline
         T & T & T & P(S)*P(C \mid S)*P(B \mid S) & 0.067575 \\ 
         \hline
         T & T & F & P(S)*P(C \mid S)*P(\lnot B \mid S) & 0.022525 \\  
         \hline
         T & F & T & P(S)*P(\lnot C \mid S)*P(B \mid S) & 0.011925 \\  
         \hline
         T & F & F & P(S)*P(\lnot C \mid S)*P(\lnot B \mid S) & 0.003975 \\  
         \hline
         F & T & T & P(\lnot S)*P(C \mid \lnot S)*P(B \mid \lnot S) & 0.03576 \\ 
         \hline
         F & T & F & P(\lnot S)*P(C \mid \lnot S)*P(\lnot B \mid \lnot S) & 0.32184 \\  
         \hline
         F & F & T & P(\lnot S)*P(\lnot C \mid \lnot S)*P(B \mid \lnot S) & 0.05364 \\ 
         \hline
         F & F & F & P(\lnot S)*P(\lnot C \mid \lnot S)*P(\lnot B \mid \lnot S) & 0.48276 \\ 
         \hline
         \hline
        \end{tabular}
        \end{center}
        \newline
        \newline
        
    \end{enumerate}
   
\end{document}